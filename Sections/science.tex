\subsection{Live Science \href{https://www.livescience.com/}{\ding{225}}}
\subsubsection{This colossal extinct whale was the heaviest animal to ever live \href{https://www.livescience.com/animals/whales/this-colossal-extinct-whale-was-the-heaviest-animal-to-ever-live}{\ding{225}}}
\textit{02-Aug-2023}

Paleontologists in Peru have described an ancient species of whale that was way heavier than a blue whale.
\subsubsection{Google's 'mind-reading' AI can tell what music you listened to based on your brain signals \href{https://www.livescience.com/health/neuroscience/googles-mind-reading-ai-can-tell-what-music-you-listened-to-based-on-your-brain-signals}{\ding{225}}}
\textit{02-Aug-2023}

Artificial intelligence can produce music that sounds similar to tunes people were listening to as they had their brains scanned, a collaborative study from Google and Osaka University shows.
\subsubsection{Save 21\% on the Jaybird Vista 2 workout earphones at Amazon \href{https://www.livescience.com/health/exercise/save-21-on-the-jaybird-vista-2-workout-earphones-at-amazon}{\ding{225}}}
\textit{02-Aug-2023}

Reduced by over \$30, you can currently get the top-rated Jaybird Vista 2 workout earphones for under \$120 at Amazon.
\subsubsection{'Black swan' pathogens from ancient permafrost may be getting ready to wake up \href{https://www.livescience.com/planet-earth/arctic/black-swan-pathogens-from-ancient-permafrost-may-be-getting-ready-to-wake-up}{\ding{225}}}
\textit{02-Aug-2023}

Scientists simulated ancient viruses to see what impact they would have on the environment. While most had few consequences, 1\% were capable of killing their hosts and disrupting ecosystems.
\subsubsection{Lego Ideas Motorized Lighthouse review \href{https://www.livescience.com/human-behavior/arts-entertainment/lego-ideas-motorized-lighthouse-review-21335}{\ding{225}}}
\textit{02-Aug-2023}

The Lego Ideas Motorized Lighthouse is both an impressive piece of Lego engineering, and one of the most visually stunning pieces they've ever made.
\subsubsection{'Exceptional' winged Medusa discovered in Roman-era mosaic in Spain \href{https://www.livescience.com/archaeology/romans/exceptional-winged-medusa-discovered-in-roman-era-mosaic-in-spain}{\ding{225}}}
\textit{02-Aug-2023}

The 1,800-year-old Medusa mosaic was discovered in the remains of lavish Roman-era house in western Spain.
\subsubsection{NASA hears 'heartbeat' signal from Voyager 2 probe a week after losing contact \href{https://www.livescience.com/space/space-exploration/nasa-hears-heartbeat-signal-from-voyager-2-probe-a-week-after-losing-contact}{\ding{225}}}
\textit{01-Aug-2023}

NASA engineers have received a "heartbeat" signal from Voyager 2, bringing hope that they may be able to reestablish contact with the probe months ahead of schedule.
\subsubsection{Siblings rapidly lost their body fat in infancy due to rare, newly described gene mutation \href{https://www.livescience.com/health/genetics/siblings-rapidly-lost-their-body-fat-in-infancy-due-to-rare-newly-described-gene-mutation}{\ding{225}}}
\textit{01-Aug-2023}

Doctors found a novel gene mutation in two siblings with the same rare disorder.
\subsubsection{Rare 'Einstein cross' warps light from one of the universe's brightest objects in this stunning image \href{https://www.livescience.com/physics-mathematics/gravity/rare-einstein-cross-warps-light-from-one-of-the-universes-brightest-objects-in-this-stunning-image}{\ding{225}}}
\textit{01-Aug-2023}

Einstein predicted the existence of these crosses back in 1915. Now, they are used to study distant galaxies.
\subsubsection{James Webb telescope discovers giant question mark galaxy in deep space \href{https://www.livescience.com/space/james-webb-telescope-discovers-giant-question-mark-galaxy-in-deep-space}{\ding{225}}}
\textit{01-Aug-2023}

The James Webb Space Telescope spied a cosmic question mark in deep space while observing two young stars located more than 1,000 light-years from Earth.
\subsection{New Scientist \href{https://www.newscientist.com/}{\ding{225}}}
\subsubsection{Your gut microbiome is linked to your fitness and biological age \href{https://www.newscientist.com/article/2385367-your-gut-microbiome-is-linked-to-your-fitness-and-biological-age/?utm_campaign=RSS%7CNSNS&utm_source=NSNS&utm_medium=RSS&utm_content=home}{\ding{225}}}
\textit{02-Aug-2023}

Altering the gut microbiome via probiotics could one day help people to maintain a certain level of fitness and reduce the rate of their biological ageing
\subsubsection{Artificial spider silk could help us harvest drinking water from air \href{https://www.newscientist.com/article/2384658-artificial-spider-silk-could-help-us-harvest-drinking-water-from-air/?utm_campaign=RSS%7CNSNS&utm_source=NSNS&utm_medium=RSS&utm_content=home}{\ding{225}}}
\textit{02-Aug-2023}

Helical bumps on artificial fibres can carry 2000 times more water than the volume of the bumps themselves, which could help people harvest drinking water from the air
\subsubsection{Ancient Babylonian arson in Jerusalem revealed by chemical clues \href{https://www.newscientist.com/article/2385449-ancient-babylonian-arson-in-jerusalem-revealed-by-chemical-clues/?utm_campaign=RSS%7CNSNS&utm_source=NSNS&utm_medium=RSS&utm_content=home}{\ding{225}}}
\textit{02-Aug-2023}

By analysing charred remains of a Jerusalem building, archaeologists have uncovered details of how it was burned down by Babylonian invaders in 586 BC
\subsubsection{Oldest adult jellyfish fossil ever found is over 500 million years old \href{https://www.newscientist.com/article/2385670-oldest-adult-jellyfish-fossil-ever-found-is-over-500-million-years-old/?utm_campaign=RSS%7CNSNS&utm_source=NSNS&utm_medium=RSS&utm_content=home}{\ding{225}}}
\textit{02-Aug-2023}

A fossil discovered in Canada reveals that jellyfish developed the swimming stage of their life cycle more than half a billion years ago
\subsubsection{NASA has regained contact with Voyager 2 after losing it for a week \href{https://www.newscientist.com/article/2385738-nasa-has-regained-contact-with-voyager-2-after-losing-it-for-a-week/?utm_campaign=RSS%7CNSNS&utm_source=NSNS&utm_medium=RSS&utm_content=home}{\ding{225}}}
\textit{01-Aug-2023}

After accidentally turning the Voyager 2 spacecraft away from Earth and losing contact with it, NASA engineers have now heard a “heartbeat signal” that shows it is still okay
\subsubsection{Something strange is happening in the Pacific and we must find out why \href{https://www.newscientist.com/article/mg25934500-100-something-strange-is-happening-in-the-pacific-and-we-must-find-out-why/?utm_campaign=RSS%7CNSNS&utm_source=NSNS&utm_medium=RSS&utm_content=home}{\ding{225}}}
\textit{01-Aug-2023}

Unexpectedly, the eastern Pacific Ocean is cooling. If this “cold tongue” continues, it could reduce greenhouse gas warming by 30 per cent – but also bring megadrought to the US
\subsubsection{Lead exposure linked to higher risk of engaging in criminal behaviour \href{https://www.newscientist.com/article/2385578-lead-exposure-linked-to-higher-risk-of-engaging-in-criminal-behaviour/?utm_campaign=RSS%7CNSNS&utm_source=NSNS&utm_medium=RSS&utm_content=home}{\ding{225}}}
\textit{01-Aug-2023}

Higher exposure to lead in childhood is associated with a greater likelihood of criminality, a review of 17 studies has found, but whether the element is causing the behaviour rather than just being associated with it is unknown
\subsubsection{Room-temperature superconductors: Here's everything you need to know \href{https://www.newscientist.com/article/2385270-room-temperature-superconductors-heres-everything-you-need-to-know/?utm_campaign=RSS%7CNSNS&utm_source=NSNS&utm_medium=RSS&utm_content=home}{\ding{225}}}
\textit{28-Jul-2023}

Truly convenient materials that can conduct electricity perfectly have long been sought after by researchers, but their history is one of difficult experiments, theoretical puzzles and scientific controversy
\subsubsection{Euclid space telescope sends back amazing first images of the cosmos \href{https://www.newscientist.com/article/2385424-euclid-space-telescope-sends-back-amazing-first-images-of-the-cosmos/?utm_campaign=RSS%7CNSNS&utm_source=NSNS&utm_medium=RSS&utm_content=home}{\ding{225}}}
\textit{31-Jul-2023}

The European Space Agency’s Euclid telescope launched on 1 July, and now it has delivered its first stunning pictures of stars and galaxies across the cosmos
\subsubsection{Earth's early warmth may be explained by methane-making reaction \href{https://www.newscientist.com/article/2385596-earths-early-warmth-may-be-explained-by-methane-making-reaction/?utm_campaign=RSS%7CNSNS&utm_source=NSNS&utm_medium=RSS&utm_content=home}{\ding{225}}}
\textit{01-Aug-2023}

A chemical process that produces methane without living organisms could have warmed up the young Earth – and may complicate the search for life elsewhere
\subsubsection{Male moths make their own perfume from flowers to attract females \href{https://www.newscientist.com/article/2385534-male-moths-make-their-own-perfume-from-flowers-to-attract-females/?utm_campaign=RSS%7CNSNS&utm_source=NSNS&utm_medium=RSS&utm_content=home}{\ding{225}}}
\textit{01-Aug-2023}

Tobacco budworm moth males gather a sweet-smelling chemical from flowers and release it from hairy appendages when they are near females to make themselves more attractive
\subsubsection{Is it possible to drill a hole straight through a planet? \href{https://www.newscientist.com/article/2385526-is-it-possible-to-drill-a-hole-straight-through-a-planet/?utm_campaign=RSS%7CNSNS&utm_source=NSNS&utm_medium=RSS&utm_content=home}{\ding{225}}}
\textit{01-Aug-2023}

Could we bore a hole through the centre of Earth? What would it be like to fling yourself through it? The Dead Planets Society podcast digs deep into the potential hazards
\subsubsection{The best new science fiction books of August 2023 \href{https://www.newscientist.com/article/2385317-the-best-new-science-fiction-books-of-august-2023/?utm_campaign=RSS%7CNSNS&utm_source=NSNS&utm_medium=RSS&utm_content=home}{\ding{225}}}
\textit{01-Aug-2023}

From speculative novellas by Josh Malerman to a first venture into science fiction from H is for Hawk author Helen Macdonald, August brims with sci-fi potential, says culture editor Alison Flood
\subsubsection{Essential molecule for life spotted floating in space for first time \href{https://www.newscientist.com/article/2384657-essential-molecule-for-life-spotted-floating-in-space-for-first-time/?utm_campaign=RSS%7CNSNS&utm_source=NSNS&utm_medium=RSS&utm_content=home}{\ding{225}}}
\textit{01-Aug-2023}

Carbonic acid, an important component of amino acids, has been seen in a cloud of gas near the centre of the Milky Way, which could help us explain how life came to be on Earth
\subsubsection{Our solar system might be 1.1 million years older than we thought \href{https://www.newscientist.com/article/2385393-our-solar-system-might-be-1-1-million-years-older-than-we-thought/?utm_campaign=RSS%7CNSNS&utm_source=NSNS&utm_medium=RSS&utm_content=home}{\ding{225}}}
\textit{01-Aug-2023}

A new analysis of small flecks in meteorites calculates the age of the solar system as 4.5684 billion years old, rather than 4.5673 billion
\subsubsection{Arctic foxes help create habitats as ‘gardeners’ of the tundra \href{https://www.newscientist.com/article/2385227-arctic-foxes-help-create-habitats-as-gardeners-of-the-tundra/?utm_campaign=RSS%7CNSNS&utm_source=NSNS&utm_medium=RSS&utm_content=home}{\ding{225}}}
\textit{01-Aug-2023}

Satellite images support the claim that Arctic foxes promote the growth of the lush vegetation seen around their dens
\subsubsection{Space germs: How can we avoid contaminating other worlds? \href{https://www.newscientist.com/article/2375987-space-germs-how-can-we-avoid-contaminating-other-worlds/?utm_campaign=RSS%7CNSNS&utm_source=NSNS&utm_medium=RSS&utm_content=home}{\ding{225}}}
\textit{09-Jun-2023}

Headlines about alien flu from outer space exaggerate the risk of extraterrestrial microbes but we still need to be careful about taking our own germs off-planet, says astrophysicist Erika Nesvold
\subsubsection{Energy-storing concrete could form foundations for solar-powered homes \href{https://www.newscientist.com/article/2385500-energy-storing-concrete-could-form-foundations-for-solar-powered-homes/?utm_campaign=RSS%7CNSNS&utm_source=NSNS&utm_medium=RSS&utm_content=home}{\ding{225}}}
\textit{31-Jul-2023}

A mixture of cement and fine charcoal can become a supercapacitor that could someday charge homes or electric vehicles
\subsubsection{Which dietary supplements actually work and which should you take \href{https://www.newscientist.com/article/mg25934500-200-which-dietary-supplements-actually-work-and-which-should-you-take/?utm_campaign=RSS%7CNSNS&utm_source=NSNS&utm_medium=RSS&utm_content=home}{\ding{225}}}
\textit{31-Jul-2023}

From vitamin D to turmeric extracts and probiotics, nutritional supplements are a booming industry. But what is the evidence behind the claimed benefits?
\subsubsection{Mathematicians find 27 tickets that guarantee UK National Lottery win \href{https://www.newscientist.com/article/2384455-mathematicians-find-27-tickets-that-guarantee-uk-national-lottery-win/?utm_campaign=RSS%7CNSNS&utm_source=NSNS&utm_medium=RSS&utm_content=home}{\ding{225}}}
\textit{28-Jul-2023}

Buying a specific set of 27 tickets for the UK National Lottery will mathematically guarantee that you win something
\subsubsection{Nightingales match the pitch of their rivals in singing duels \href{https://www.newscientist.com/article/2385379-nightingales-match-the-pitch-of-their-rivals-in-singing-duels/?utm_campaign=RSS%7CNSNS&utm_source=NSNS&utm_medium=RSS&utm_content=home}{\ding{225}}}
\textit{31-Jul-2023}

Male nightingales respond to each other’s songs by whistling back at the same pitch when competing for territory, and they even copy the pitch of artificial whistle sounds
\subsubsection{What will the UK push for more North Sea oil and gas really achieve? \href{https://www.newscientist.com/article/2385452-what-will-the-uk-push-for-more-north-sea-oil-and-gas-really-achieve/?utm_campaign=RSS%7CNSNS&utm_source=NSNS&utm_medium=RSS&utm_content=home}{\ding{225}}}
\textit{31-Jul-2023}

The UK’s plan to issue new oil and gas licences will make a minimal impact on North Sea production levels – but it will dent the country's reputation as a climate leader
\subsubsection{AI can make life easier, but it could seriously distort democracies \href{https://www.newscientist.com/article/mg25934493-600-ai-can-make-life-easier-but-it-could-seriously-distort-democracies/?utm_campaign=RSS%7CNSNS&utm_source=NSNS&utm_medium=RSS&utm_content=home}{\ding{225}}}
\textit{26-Jul-2023}

While artificial intelligence is a transformative tool that can release us from tedious tasks, we must be wary of its unprecedented power to to supercharge propaganda
\subsubsection{Swimming behind someone cuts drag by up to 40 per cent \href{https://www.newscientist.com/article/2385153-swimming-behind-someone-cuts-drag-by-up-to-40-per-cent/?utm_campaign=RSS%7CNSNS&utm_source=NSNS&utm_medium=RSS&utm_content=home}{\ding{225}}}
\textit{31-Jul-2023}

Swimmers in open water races can reduce the amount of drag forces they experience by swimming behind or at the hip level of another swimmer
\subsubsection{Wild bees are rapidly shrinking due to global warming \href{https://www.newscientist.com/article/2384886-wild-bees-are-rapidly-shrinking-due-to-global-warming/?utm_campaign=RSS%7CNSNS&utm_source=NSNS&utm_medium=RSS&utm_content=home}{\ding{225}}}
\textit{31-Jul-2023}

Bees in a well-preserved Spanish wilderness weigh less than they did decades ago, possibly because rising temperatures are affecting their development and food
\subsubsection{How to make perfect tempura at home \href{https://www.newscientist.com/article/mg25934491-600-how-to-make-perfect-tempura-at-home/?utm_campaign=RSS%7CNSNS&utm_source=NSNS&utm_medium=RSS&utm_content=home}{\ding{225}}}
\textit{26-Jul-2023}

From using less protein to adding ethanol, these science-backed tricks can help you make the crispiest tempura
\subsubsection{Can AI ever become conscious and how would we know if that happens? \href{https://www.newscientist.com/article/2384077-can-ai-ever-become-conscious-and-how-would-we-know-if-that-happens/?utm_campaign=RSS%7CNSNS&utm_source=NSNS&utm_medium=RSS&utm_content=home}{\ding{225}}}
\textit{25-Jul-2023}

It sounds far-fetched, but researchers are trying to recreate subjective experience in AIs, even if disagreement over what consciousness is will make it difficult to test
\subsubsection{Aliens: Dark Descent review: Great game finally joins the franchise \href{https://www.newscientist.com/article/mg25934491-200-aliens-dark-descent-review-great-game-finally-joins-the-franchise/?utm_campaign=RSS%7CNSNS&utm_source=NSNS&utm_medium=RSS&utm_content=home}{\ding{225}}}
\textit{26-Jul-2023}

Based on the first film sequel, Aliens: Dark Descent sees you in charge of exploring a moon in search of hostile creatures. Things can get real bad, real quick, discovers Jacob Aron
\subsubsection{The slippery quest to invent a safer, more effective sunscreen \href{https://www.newscientist.com/article/mg25934491-500-the-slippery-quest-to-invent-a-safer-more-effective-sunscreen/?utm_campaign=RSS%7CNSNS&utm_source=NSNS&utm_medium=RSS&utm_content=home}{\ding{225}}}
\textit{24-Jul-2023}

Concerns over chemical sunscreens harming coral reefs and maybe even our health are inspiring a new generation of sun lotions that would offer greater protection that endures
\subsubsection{These are the five most shocking climate records this week \href{https://www.newscientist.com/article/2384931-these-are-the-five-most-shocking-climate-records-this-week/?utm_campaign=RSS%7CNSNS&utm_source=NSNS&utm_medium=RSS&utm_content=home}{\ding{225}}}
\textit{28-Jul-2023}

From "hot tub" sea temperatures in Florida to wildfires across Greek islands, we are seeing a record-breaking season of climate extremes
\subsubsection{The biggest scientific challenges that AI is already helping to crack \href{https://www.newscientist.com/article/2384085-the-biggest-scientific-challenges-that-ai-is-already-helping-to-crack/?utm_campaign=RSS%7CNSNS&utm_source=NSNS&utm_medium=RSS&utm_content=home}{\ding{225}}}
\textit{25-Jul-2023}

AI isn't just for chatbots – many companies are using it to tackle everything from protein folding and drug development to commercially viable nuclear fusion
\subsubsection{Rats have a 'laugh centre' in their brains that responds to tickling \href{https://www.newscientist.com/article/2385124-rats-have-a-laugh-centre-in-their-brains-that-responds-to-tickling/?utm_campaign=RSS%7CNSNS&utm_source=NSNS&utm_medium=RSS&utm_content=home}{\ding{225}}}
\textit{28-Jul-2023}

A region of the midbrain in rodents appears to be instrumental in enabling rats to engage in laughter and play
\subsubsection{Women may age fastest during their 30s and 50s \href{https://www.newscientist.com/article/2384445-women-may-age-fastest-during-their-30s-and-50s/?utm_campaign=RSS%7CNSNS&utm_source=NSNS&utm_medium=RSS&utm_content=home}{\ding{225}}}
\textit{28-Jul-2023}

A woman may be most likely to give birth in her 30s and go through the menopause in her 50s, with these life events causing hormonal changes that may accelerate ageing
\subsubsection{Intermittent fasting could boost immunity in addition to weight loss \href{https://www.newscientist.com/article/2385137-intermittent-fasting-could-boost-immunity-in-addition-to-weight-loss/?utm_campaign=RSS%7CNSNS&utm_source=NSNS&utm_medium=RSS&utm_content=home}{\ding{225}}}
\textit{28-Jul-2023}

Immune cells are more effective at fending off pathogens when they use ketones – which are produced during fasting – rather than glucose for energy
\subsubsection{Loss of smell may be an early sign of Alzheimer's in at-risk people \href{https://www.newscientist.com/article/2384964-loss-of-smell-may-be-an-early-sign-of-alzheimers-in-at-risk-people/?utm_campaign=RSS%7CNSNS&utm_source=NSNS&utm_medium=RSS&utm_content=home}{\ding{225}}}
\textit{28-Jul-2023}

People who carry a genetic variant that raises the risk of Alzheimer's disease may start to lose their sense of smell before they experience any decline in cognitive function
\subsubsection{AI news recap: While Hollywood strikes, is ChatGPT getting worse? \href{https://www.newscientist.com/article/2384188-ai-news-recap-while-hollywood-strikes-is-chatgpt-getting-worse/?utm_campaign=RSS%7CNSNS&utm_source=NSNS&utm_medium=RSS&utm_content=home}{\ding{225}}}
\textit{28-Jul-2023}

There is anger over a Netflix AI job paying up to \$900,000, coming as actors are still striking over the use of AI in film and TV.  In other AI news, problems with training data can cause glitches or make chatbots more racist
\subsubsection{Ancient make-up in Chinese tomb includes concealer and skin lightener \href{https://www.newscientist.com/article/2384091-ancient-make-up-in-chinese-tomb-includes-concealer-and-skin-lightener/?utm_campaign=RSS%7CNSNS&utm_source=NSNS&utm_medium=RSS&utm_content=home}{\ding{225}}}
\textit{28-Jul-2023}

Two cosmetic products have been identified from residues in the tomb of a non-noble woman who lived in China 2000 years ago, suggesting the widespread use of make-up
\subsubsection{Shirt woven with metal could help medical implants communicate \href{https://www.newscientist.com/article/2384486-shirt-woven-with-metal-could-help-medical-implants-communicate/?utm_campaign=RSS%7CNSNS&utm_source=NSNS&utm_medium=RSS&utm_content=home}{\ding{225}}}
\textit{28-Jul-2023}

Transmitting data around the body can be tricky, but a shirt with a metal upgrade could help
\subsubsection{Striking photo of lone tree is stark warning about Bolivia’s future \href{https://www.newscientist.com/article/mg25934490-800-striking-photo-of-lone-tree-is-stark-warning-about-bolivias-future/?utm_campaign=RSS%7CNSNS&utm_source=NSNS&utm_medium=RSS&utm_content=home}{\ding{225}}}
\textit{26-Jul-2023}

Bolivia's deforestation crisis is underlined in this set of images from Terraforming, a project by photographer Matjaž Krivic and journalist Maja Prijatelj Videmšek
\subsubsection{In Light-Years There's No Hurry review: Embracing a cosmic perspective \href{https://www.newscientist.com/article/mg25934491-000-in-light-years-theres-no-hurry-review-embracing-a-cosmic-perspective/?utm_campaign=RSS%7CNSNS&utm_source=NSNS&utm_medium=RSS&utm_content=home}{\ding{225}}}
\textit{26-Jul-2023}

A charming, challenging book argues that we can improve our well-being by tapping into the spiritually transforming "overview effect" that astronauts report after seeing earth from space
\subsubsection{Formula E unveils racing car made from electronic waste \href{https://www.newscientist.com/article/2384959-formula-e-unveils-racing-car-made-from-electronic-waste/?utm_campaign=RSS%7CNSNS&utm_source=NSNS&utm_medium=RSS&utm_content=home}{\ding{225}}}
\textit{28-Jul-2023}

A Formula E racing car made from iPhones, chargers, batteries and single-use vapes has been built in the UK to highlight the increasing problem of electronic waste
\subsubsection{Targeted mRNA delivery will lead to cheaper cures for many diseases \href{https://www.newscientist.com/article/2384828-targeted-mrna-delivery-will-lead-to-cheaper-cures-for-many-diseases/?utm_campaign=RSS%7CNSNS&utm_source=NSNS&utm_medium=RSS&utm_content=home}{\ding{225}}}
\textit{27-Jul-2023}

A technique for delivering mRNAs to blood stem cells should enable better and cheaper treatments for conditions from sickle cell disease to ageing
\subsubsection{People in the US passed swine flu to pigs nearly 400 times in 12 years \href{https://www.newscientist.com/article/2385056-people-in-the-us-passed-swine-flu-to-pigs-nearly-400-times-in-12-years/?utm_campaign=RSS%7CNSNS&utm_source=NSNS&utm_medium=RSS&utm_content=home}{\ding{225}}}
\textit{27-Jul-2023}

A particular strain of swine flu was first recorded in people in 2009. Since then, humans have passed the strain to pigs at least 370 times in the US
\subsubsection{Origin of Indo-European languages traced back to 8000 years ago \href{https://www.newscientist.com/article/2385057-origin-of-indo-european-languages-traced-back-to-8000-years-ago/?utm_campaign=RSS%7CNSNS&utm_source=NSNS&utm_medium=RSS&utm_content=home}{\ding{225}}}
\textit{27-Jul-2023}

An analysis of related words in 161 languages suggests their shared roots lie in the Middle East – a conclusion that also fits with DNA evidence
\subsubsection{Is geological hydrogen a green solution to our energy needs? \href{https://www.newscientist.com/article/2384933-is-geological-hydrogen-a-green-solution-to-our-energy-needs/?utm_campaign=RSS%7CNSNS&utm_source=NSNS&utm_medium=RSS&utm_content=home}{\ding{225}}}
\textit{27-Jul-2023}

Plans to extract hydrogen from underground reservoirs have been presented as climate-friendly, but it is still unclear just how much they could increase global warming
\subsubsection{Supersonic cracks seem to be breaking the laws of physics \href{https://www.newscientist.com/article/2384843-supersonic-cracks-seem-to-be-breaking-the-laws-of-physics/?utm_campaign=RSS%7CNSNS&utm_source=NSNS&utm_medium=RSS&utm_content=home}{\ding{225}}}
\textit{27-Jul-2023}

An experiment with elastic gels broke the theoretical speed limit for how fast cracks can move through materials, raising new questions about the physics of fractures
\subsubsection{UFO hearing: Why do so many people believe aliens have visited Earth? \href{https://www.newscientist.com/article/2384991-ufo-hearing-why-do-so-many-people-believe-aliens-have-visited-earth/?utm_campaign=RSS%7CNSNS&utm_source=NSNS&utm_medium=RSS&utm_content=home}{\ding{225}}}
\textit{27-Jul-2023}

Despite testimony by David Grusch to US Congress about "non-human biologics" and UFO crash sites, there is still no evidence aliens have ever come to Earth. Why are people taking such claims seriously, asks Jacob Aron
\subsubsection{Young mouse blood extends lives of older ones while rejuvenating them \href{https://www.newscientist.com/article/2384878-young-mouse-blood-extends-lives-of-older-ones-while-rejuvenating-them/?utm_campaign=RSS%7CNSNS&utm_source=NSNS&utm_medium=RSS&utm_content=home}{\ding{225}}}
\textit{27-Jul-2023}

Surgically attaching old mice to young mice to exchange their blood has previously been shown to rejuvenate the older individuals' brains, livers and muscles. Now, it has been shown to also extend their lifespan, even after the animals have been detached
\subsubsection{Stars have an innate twinkle – and now you can listen to it \href{https://www.newscientist.com/article/2384861-stars-have-an-innate-twinkle-and-now-you-can-listen-to-it/?utm_campaign=RSS%7CNSNS&utm_source=NSNS&utm_medium=RSS&utm_content=home}{\ding{225}}}
\textit{27-Jul-2023}

Simulations of the rippling that occurs inside stars has made it possible to turn this innate twinkling into audio
\subsubsection{Could Elon Musk's xAI be exactly what the world needs? \href{https://www.newscientist.com/article/mg25934490-700-could-elon-musks-xai-be-exactly-what-the-world-needs/?utm_campaign=RSS%7CNSNS&utm_source=NSNS&utm_medium=RSS&utm_content=home}{\ding{225}}}
\textit{26-Jul-2023}

As academics struggle to compete with private investment, perhaps Musk’s new artificial intelligence venture really can tackle the “true nature of the universe”
\subsubsection{River pollution 'offsets' for homes in England and Wales may not work \href{https://www.newscientist.com/article/2384196-river-pollution-offsets-for-homes-in-england-and-wales-may-not-work/?utm_campaign=RSS%7CNSNS&utm_source=NSNS&utm_medium=RSS&utm_content=home}{\ding{225}}}
\textit{27-Jul-2023}

Some home builders in England and Wales are allowed to buy nutrient credits to "offset" the pollution caused by new houses, but their efficacy is in doubt
\subsubsection{We should all be curbing our water usage – before it's too late \href{https://www.newscientist.com/article/mg25934493-500-we-should-all-be-curbing-our-water-usage-before-its-too-late/?utm_campaign=RSS%7CNSNS&utm_source=NSNS&utm_medium=RSS&utm_content=home}{\ding{225}}}
\textit{26-Jul-2023}

Many have an attitude towards water that assumes it will always be plentiful. But unless people, governments and corporations change their ways, the demand may soon outstrip the supply, says Jason Arunn Murugesu
\subsubsection{We may have finally figured out how galaxy-scale magnetic fields arose \href{https://www.newscientist.com/article/2384429-we-may-have-finally-figured-out-how-galaxy-scale-magnetic-fields-arose/?utm_campaign=RSS%7CNSNS&utm_source=NSNS&utm_medium=RSS&utm_content=home}{\ding{225}}}
\textit{27-Jul-2023}

Large-scale magnetic fields that fill up the universe may have grown from tiny magnetic fields that arose spontaneously in turbulent plasmas
\subsubsection{How cancer-fighting immune cells could be made safer and more powerful \href{https://www.newscientist.com/article/2383957-how-cancer-fighting-immune-cells-could-be-made-safer-and-more-powerful/?utm_campaign=RSS%7CNSNS&utm_source=NSNS&utm_medium=RSS&utm_content=home}{\ding{225}}}
\textit{27-Jul-2023}

Engineered immune cells called CAR T-cells are highly effective against cancer but they are also dangerous – but an upgrade could make them safer and more effective
\subsubsection{What an Owl Knows review: Inside the world of this mysterious bird \href{https://www.newscientist.com/article/mg25934490-900-what-an-owl-knows-review-inside-the-world-of-this-mysterious-bird/?utm_campaign=RSS%7CNSNS&utm_source=NSNS&utm_medium=RSS&utm_content=home}{\ding{225}}}
\textit{26-Jul-2023}

Owls are "in our DNA" says Jennifer Ackerman, the author of a new book that investigates the latest scientific and archaeological findings about the enigmatic birds
\subsubsection{Cannabis poisoning cases quadruple in children after legalisation \href{https://www.newscientist.com/article/2384800-cannabis-poisoning-cases-quadruple-in-children-after-legalisation/?utm_campaign=RSS%7CNSNS&utm_source=NSNS&utm_medium=RSS&utm_content=home}{\ding{225}}}
\textit{27-Jul-2023}

The risk of cannabis poisoning in children increased fourfold after the drug was legalised for medical or recreational use in different locations, primarily due to edibles
\subsubsection{Room-temperature superconductor 'breakthrough' met with scepticism \href{https://www.newscientist.com/article/2384782-room-temperature-superconductor-breakthrough-met-with-scepticism/?utm_campaign=RSS%7CNSNS&utm_source=NSNS&utm_medium=RSS&utm_content=home}{\ding{225}}}
\textit{26-Jul-2023}

Creating a material that perfectly conducts electricity at room temperature and pressure would be a big deal, but a research team's claims of creating one has attracted more scrutiny than optimism
\subsubsection{Lizard puts less effort into wooing and choosing mates when it's hot \href{https://www.newscientist.com/article/2384434-lizard-puts-less-effort-into-wooing-and-choosing-mates-when-its-hot/?utm_campaign=RSS%7CNSNS&utm_source=NSNS&utm_medium=RSS&utm_content=home}{\ding{225}}}
\textit{26-Jul-2023}

Spiny lava lizards spent less time wooing and selecting a sexual partner when exposed to temperatures that are warmer than usual
\subsubsection{Wood-munching fungi can break down common type of plastic \href{https://www.newscientist.com/article/2384693-wood-munching-fungi-can-break-down-common-type-of-plastic/?utm_campaign=RSS%7CNSNS&utm_source=NSNS&utm_medium=RSS&utm_content=home}{\ding{225}}}
\textit{26-Jul-2023}

Fungi isolated from rotting hardwood trees can break down sheets of low-density polyethylene, one of the most abundant plastics on Earth
\subsubsection{How to use AI to make your life simpler, cheaper and more productive \href{https://www.newscientist.com/article/2384092-how-to-use-ai-to-make-your-life-simpler-cheaper-and-more-productive/?utm_campaign=RSS%7CNSNS&utm_source=NSNS&utm_medium=RSS&utm_content=home}{\ding{225}}}
\textit{25-Jul-2023}

From helping you to craft the perfect email to providing personal training and meal planning, a whole host of generative AI tools are here to streamline your daily grind
\subsubsection{Magnetic-levitation device separates and sorts viruses from the air \href{https://www.newscientist.com/article/2384112-magnetic-levitation-device-separates-and-sorts-viruses-from-the-air/?utm_campaign=RSS%7CNSNS&utm_source=NSNS&utm_medium=RSS&utm_content=home}{\ding{225}}}
\textit{26-Jul-2023}

Viruses and bacteria floating in the air can be sorted for further analysis using magnets and the metal gadolinium
\subsubsection{Aspirin raises the risk of brain bleeds and may not prevent strokes \href{https://www.newscientist.com/article/2384633-aspirin-raises-the-risk-of-brain-bleeds-and-may-not-prevent-strokes/?utm_campaign=RSS%7CNSNS&utm_source=NSNS&utm_medium=RSS&utm_content=home}{\ding{225}}}
\textit{26-Jul-2023}

Doctors sometimes recommend older people take a low dose of aspirin to reduce their risk of the most common type of stroke, but a study suggests this is no more effective than a placebo and raises the risk of brain bleeds
\subsubsection{The start of spring in the Arctic is increasingly unpredictable \href{https://www.newscientist.com/article/2384583-the-start-of-spring-in-the-arctic-is-increasingly-unpredictable/?utm_campaign=RSS%7CNSNS&utm_source=NSNS&utm_medium=RSS&utm_content=home}{\ding{225}}}
\textit{26-Jul-2023}

Instead of coming earlier and earlier as the climate warms, the onset of spring in the Arctic is now extremely variable from year to year, bringing challenges to wildlife
\subsubsection{Driverless cars could get AI-powered heat vision for nighttime driving \href{https://www.newscientist.com/article/2384435-driverless-cars-could-get-ai-powered-heat-vision-for-nighttime-driving/?utm_campaign=RSS%7CNSNS&utm_source=NSNS&utm_medium=RSS&utm_content=home}{\ding{225}}}
\textit{26-Jul-2023}

Self-driving vehicles that struggle with recognising objects at night could get a boost from a heat-assisted detection and ranging system
\subsubsection{Best science fiction films about space, according to an astrophysicist \href{https://www.newscientist.com/article/2384184-best-science-fiction-films-about-space-according-to-an-astrophysicist/?utm_campaign=RSS%7CNSNS&utm_source=NSNS&utm_medium=RSS&utm_content=home}{\ding{225}}}
\textit{25-Jul-2023}

From 2001: A Space Odyssey to WALL-E, Erika Nesvold picks her favourite sci-fi films set in space (and explains why E.T. missed out)
\subsubsection{Rethinking reality: Is the entire universe a single quantum object? \href{https://www.newscientist.com/article/mg25834460-800-rethinking-reality-is-the-entire-universe-a-single-quantum-object/?utm_campaign=RSS%7CNSNS&utm_source=NSNS&utm_medium=RSS&utm_content=home}{\ding{225}}}
\textit{05-Jul-2023}

In the face of new evidence, physicists are starting to view the cosmos not as made up of disparate layers, but as a quantum whole linked by entanglement
\subsubsection{The best new science fiction books of July 2023 \href{https://www.newscientist.com/article/2380747-the-best-new-science-fiction-books-of-july-2023/?utm_campaign=RSS%7CNSNS&utm_source=NSNS&utm_medium=RSS&utm_content=home}{\ding{225}}}
\textit{03-Jul-2023}

From George R. R. Martin’s new Wild Cards anthology to Nana Kwame Adjei-Brenyah's dystopian take on America, there is a wealth of exciting science fiction out this month. Culture editor Alison Flood shares the novels she is most anticipating
\subsubsection{How hacking your metabolism can help you burn fat and prevent disease \href{https://www.newscientist.com/article/mg25634071-000-how-hacking-your-metabolism-can-help-you-burn-fat-and-prevent-disease/?utm_campaign=RSS%7CNSNS&utm_source=NSNS&utm_medium=RSS&utm_content=home}{\ding{225}}}
\textit{04-Oct-2022}

Hacking your metabolism to help your body burn fats and carbs more efficiently may be key to helping you lose weight, run for longer and reduce the risk of conditions like type two diabetes
\subsubsection{Honey made by ants could treat some bacterial and fungal infections \href{https://www.newscientist.com/article/2384459-honey-made-by-ants-could-treat-some-bacterial-and-fungal-infections/?utm_campaign=RSS%7CNSNS&utm_source=NSNS&utm_medium=RSS&utm_content=home}{\ding{225}}}
\textit{26-Jul-2023}

An ant species in Australia makes honey that killed some bacterial and fungal infections in the lab, raising hopes that its properties could be used in new drugs
\subsubsection{Vital Atlantic Ocean current could collapse as soon as 2025 \href{https://www.newscientist.com/article/2384094-vital-atlantic-ocean-current-could-collapse-as-soon-as-2025/?utm_campaign=RSS%7CNSNS&utm_source=NSNS&utm_medium=RSS&utm_content=home}{\ding{225}}}
\textit{25-Jul-2023}

A study warns that the Atlantic meridional overturning circulation is close to a tipping point that would severely disrupt the climate – but other researchers say the timing is impossible to predict
\subsubsection{Newly discovered dinosaur roamed South-East Asia 200 million years ago \href{https://www.newscientist.com/article/2383575-newly-discovered-dinosaur-roamed-south-east-asia-200-million-years-ago/?utm_campaign=RSS%7CNSNS&utm_source=NSNS&utm_medium=RSS&utm_content=home}{\ding{225}}}
\textit{26-Jul-2023}

Fossils unearthed in Thailand have been identified as a new species of dinosaur that fed on plants and roamed the wilds of South-East Asia
\subsubsection{Unique egg patterns help drongos avoid getting duped by cuckoos \href{https://www.newscientist.com/article/2384258-unique-egg-patterns-help-drongos-avoid-getting-duped-by-cuckoos/?utm_campaign=RSS%7CNSNS&utm_source=NSNS&utm_medium=RSS&utm_content=home}{\ding{225}}}
\textit{26-Jul-2023}

Fork-tailed drongos produce individualised patterns on their eggs, which may help them recognise their own and reject 94 per cent of cuckoo eggs
\subsubsection{Water seen in young planet system shows Earth may have always been wet \href{https://www.newscientist.com/article/2384405-water-seen-in-young-planet-system-shows-earth-may-have-always-been-wet/?utm_campaign=RSS%7CNSNS&utm_source=NSNS&utm_medium=RSS&utm_content=home}{\ding{225}}}
\textit{25-Jul-2023}

The James Webb Space Telescope has spotted water vapour in an area where planets may be forming, which could present an answer to the debate over how Earth got its water
\subsubsection{Building things with wood may not be as climate-friendly as thought \href{https://www.newscientist.com/article/2384394-building-things-with-wood-may-not-be-as-climate-friendly-as-thought/?utm_campaign=RSS%7CNSNS&utm_source=NSNS&utm_medium=RSS&utm_content=home}{\ding{225}}}
\textit{25-Jul-2023}

Wood is a versatile construction material that could be used to replace carbon-intensive steel and concrete in construction, however the emissions involved may have been underestimated
\subsubsection{Forget human extinction – these are the real risks posed by AI today \href{https://www.newscientist.com/article/2384063-forget-human-extinction-these-are-the-real-risks-posed-by-ai-today/?utm_campaign=RSS%7CNSNS&utm_source=NSNS&utm_medium=RSS&utm_content=home}{\ding{225}}}
\textit{25-Jul-2023}

Amid warnings that advanced AI could wipe out humanity, some experts insist we should be more worried about people using existing AIs to supercharge the spread of misinformation
\subsubsection{What generative AI really means for the economy, jobs and education \href{https://www.newscientist.com/article/2384034-what-generative-ai-really-means-for-the-economy-jobs-and-education/?utm_campaign=RSS%7CNSNS&utm_source=NSNS&utm_medium=RSS&utm_content=home}{\ding{225}}}
\textit{25-Jul-2023}

Even with the capabilities they have today, the new generation of AIs will profoundly reshape the world, and your life, over the next decade. Here’s how
\subsubsection{The science and side effects of the drugs Ozempic and Wegovy \href{https://www.newscientist.com/article/2371780-the-science-and-side-effects-of-the-drugs-ozempic-and-wegovy/?utm_campaign=RSS%7CNSNS&utm_source=NSNS&utm_medium=RSS&utm_content=home}{\ding{225}}}
\textit{25-Jul-2023}

From how well they work to side effects such as hair loss, here’s the skinny on new weight loss injections that work by blocking a hormone that normally reduces appetite
\subsubsection{How does ChatGPT work and do AI-powered chatbots “think” like us? \href{https://www.newscientist.com/article/2384030-how-does-chatgpt-work-and-do-ai-powered-chatbots-think-like-us/?utm_campaign=RSS%7CNSNS&utm_source=NSNS&utm_medium=RSS&utm_content=home}{\ding{225}}}
\textit{25-Jul-2023}

The large language models behind the new chatbots are trained to predict which words are most likely to appear together – but “emergent abilities” suggest they might be doing more than that
\subsubsection{Sea level may have been higher than it is now just 6000 years ago \href{https://www.newscientist.com/article/2383478-sea-level-may-have-been-higher-than-it-is-now-just-6000-years-ago/?utm_campaign=RSS%7CNSNS&utm_source=NSNS&utm_medium=RSS&utm_content=home}{\ding{225}}}
\textit{25-Jul-2023}

Climate researchers thought that current global average sea levels were the highest in more than 100,000 years, but new models suggest oceans just 6000 years ago may have been higher than at the beginning of the industrial revolution, and possibly even higher than today
\subsubsection{Stunning photo of a young star hints how Jupiter-like planets form \href{https://www.newscientist.com/article/2384275-stunning-photo-of-a-young-star-hints-how-jupiter-like-planets-form/?utm_campaign=RSS%7CNSNS&utm_source=NSNS&utm_medium=RSS&utm_content=home}{\ding{225}}}
\textit{25-Jul-2023}

This mesmerising image of star V960 Mon spitting out gas jets to create arms larger than our entire solar system shows how massive planets may come together via gravitational instability
\subsubsection{Oppenheimer: What led to the physicist's downfall? \href{https://www.newscientist.com/article/2384144-oppenheimer-what-led-to-the-physicists-downfall/?utm_campaign=RSS%7CNSNS&utm_source=NSNS&utm_medium=RSS&utm_content=home}{\ding{225}}}
\textit{25-Jul-2023}

J. Robert Oppenheimer was instrumental in creating the first atomic bomb but afterwards spent decades campaigning against it. Christopher Nolan’s new film focuses on these later years
\subsubsection{July’s heatwave was made 1000 times more likely by climate change \href{https://www.newscientist.com/article/2384130-julys-heatwave-was-made-1000-times-more-likely-by-climate-change/?utm_campaign=RSS%7CNSNS&utm_source=NSNS&utm_medium=RSS&utm_content=home}{\ding{225}}}
\textit{25-Jul-2023}

Climate change made the heatwaves in North America and Europe at least 1000 times more likely and the heatwave in China around 50 times more likely
\subsubsection{Hearing aids reduce cognitive decline for people at risk of dementia \href{https://www.newscientist.com/article/2383792-hearing-aids-reduce-cognitive-decline-for-people-at-risk-of-dementia/?utm_campaign=RSS%7CNSNS&utm_source=NSNS&utm_medium=RSS&utm_content=home}{\ding{225}}}
\textit{25-Jul-2023}

Wearing a hearing aid didn't reduce the rate of cognitive decline among a group of older people when compared with just receiving general health advice, but it did have an effect when the researchers focused on those who are particularly at risk of dementia
\subsubsection{Some people are aware during CPR and can remember the experience \href{https://www.newscientist.com/article/2383847-some-people-are-aware-during-cpr-and-can-remember-the-experience/?utm_campaign=RSS%7CNSNS&utm_source=NSNS&utm_medium=RSS&utm_content=home}{\ding{225}}}
\textit{25-Jul-2023}

In the most detailed study yet of awareness during cardiopulmonary resuscitation, nearly half those who survived reported some kind of hazy memories, dreams or perceptions
\subsubsection{Huge amounts of plastic from artificial grass end up in the sea \href{https://www.newscientist.com/article/2383869-huge-amounts-of-plastic-from-artificial-grass-end-up-in-the-sea/?utm_campaign=RSS%7CNSNS&utm_source=NSNS&utm_medium=RSS&utm_content=home}{\ding{225}}}
\textit{25-Jul-2023}

Fibres from artificial grass make up 15 per cent of plastic pieces found in samples of seawater near Barcelona
\subsubsection{GPT-4: Is the AI behind ChatGPT getting worse? \href{https://www.newscientist.com/article/2383850-gpt-4-is-the-ai-behind-chatgpt-getting-worse/?utm_campaign=RSS%7CNSNS&utm_source=NSNS&utm_medium=RSS&utm_content=home}{\ding{225}}}
\textit{24-Jul-2023}

The AI that powers ChatGPT appears to be performing less well at mathematical problems than it was just a few months ago
\subsubsection{Your hands are probably about twice as heavy as you think they are \href{https://www.newscientist.com/article/2383998-your-hands-are-probably-about-twice-as-heavy-as-you-think-they-are/?utm_campaign=RSS%7CNSNS&utm_source=NSNS&utm_medium=RSS&utm_content=home}{\ding{225}}}
\textit{24-Jul-2023}

Experiments show that people consistently underestimate the weight of their own hands – a perceptual quirk that could be important for designing prosthetics
\subsubsection{Psychedelics show promise for treating anorexia in early trials \href{https://www.newscientist.com/article/2381646-psychedelics-show-promise-for-treating-anorexia-in-early-trials/?utm_campaign=RSS%7CNSNS&utm_source=NSNS&utm_medium=RSS&utm_content=home}{\ding{225}}}
\textit{24-Jul-2023}

Small studies suggest that psilocybin, the psychoactive compound in magic mushrooms, may reduce the severity of eating disorders and increase people’s motivation to recover from the condition
\subsubsection{Most plant-based milks have less protein and calcium than cow's milk \href{https://www.newscientist.com/article/2383908-most-plant-based-milks-have-less-protein-and-calcium-than-cows-milk/?utm_campaign=RSS%7CNSNS&utm_source=NSNS&utm_medium=RSS&utm_content=home}{\ding{225}}}
\textit{24-Jul-2023}

Plant-based milks made from almonds, oats, rice and soya beans generally contain fewer nutrients than cow's milk
\subsubsection{8 healthy habits linked to living decades longer \href{https://www.newscientist.com/article/2383669-8-healthy-habits-linked-to-living-decades-longer/?utm_campaign=RSS%7CNSNS&utm_source=NSNS&utm_medium=RSS&utm_content=home}{\ding{225}}}
\textit{24-Jul-2023}

A study of more than 700,000 people found that adopting eight healthy habits by age 40 could extend life expectancy by more than two decades
\subsubsection{Why is Twitter becoming X and should you move to Threads or Bluesky? \href{https://www.newscientist.com/article/2383886-why-is-twitter-becoming-x-and-should-you-move-to-threads-or-bluesky/?utm_campaign=RSS%7CNSNS&utm_source=NSNS&utm_medium=RSS&utm_content=home}{\ding{225}}}
\textit{24-Jul-2023}

Elon Musk is once more plunging Twitter into turmoil, this time by changing its very identity
\subsubsection{Fears of record-breaking El Niño event this year raise climate alarms \href{https://www.newscientist.com/article/2383666-fears-of-record-breaking-el-nino-event-this-year-raise-climate-alarms/?utm_campaign=RSS%7CNSNS&utm_source=NSNS&utm_medium=RSS&utm_content=home}{\ding{225}}}
\textit{24-Jul-2023}

We don't yet know how strong the developing El Niño climate pattern will be, but even a weak one risks severe global disruption
\subsubsection{Rare flower makes fake bee blood to lure pollinating insects \href{https://www.newscientist.com/article/2383731-rare-flower-makes-fake-bee-blood-to-lure-pollinating-insects/?utm_campaign=RSS%7CNSNS&utm_source=NSNS&utm_medium=RSS&utm_content=home}{\ding{225}}}
\textit{24-Jul-2023}

Jackal flies usually feed on dead bees, but a South African plant entices them by smelling like bee blood and producing a nutritious fluid
\subsubsection{Cracking consciousness will never be easy but we are making strides \href{https://www.newscientist.com/article/mg25934483-200-cracking-consciousness-will-never-be-easy-but-we-are-making-strides/?utm_campaign=RSS%7CNSNS&utm_source=NSNS&utm_medium=RSS&utm_content=home}{\ding{225}}}
\textit{19-Jul-2023}

A new way to understand where consciousness comes from and novel insights into subjective thought show that the hard problem of consciousness is worth persevering with
\subsubsection{Earth is coated in ancient space dust that could be from the moon \href{https://www.newscientist.com/article/2383756-earth-is-coated-in-ancient-space-dust-that-could-be-from-the-moon/?utm_campaign=RSS%7CNSNS&utm_source=NSNS&utm_medium=RSS&utm_content=home}{\ding{225}}}
\textit{24-Jul-2023}

A 33-million-year-old layer of Earth's crust is laced with helium-3, which is normally only found in space. Now we might have an explanation for how it got there
\subsubsection{Why putting broken pottery in your plant pot won't help with drainage \href{https://www.newscientist.com/article/mg25934481-100-why-putting-broken-pottery-in-your-plant-pot-wont-help-with-drainage/?utm_campaign=RSS%7CNSNS&utm_source=NSNS&utm_medium=RSS&utm_content=home}{\ding{225}}}
\textit{19-Jul-2023}

Putting a layer of "crocks", or broken pottery, in the base of flower pots won't improve drainage – and may actually worsen it, says James Wong
\subsubsection{Revealed: What your thoughts look like and how they compare to others’ \href{https://www.newscientist.com/article/mg25934484-800-revealed-what-your-thoughts-look-like-and-how-they-compare-to-others/?utm_campaign=RSS%7CNSNS&utm_source=NSNS&utm_medium=RSS&utm_content=home}{\ding{225}}}
\textit{19-Jul-2023}

We finally have a grasp on the many different ways of thinking and how your inner mindscape affects your experience of reality
\subsubsection{Your genes may influence how much fruit, fish or salt you eat \href{https://www.newscientist.com/article/2383747-your-genes-may-influence-how-much-fruit-fish-or-salt-you-eat/?utm_campaign=RSS%7CNSNS&utm_source=NSNS&utm_medium=RSS&utm_content=home}{\ding{225}}}
\textit{22-Jul-2023}

Nearly 500 regions of the human genome appear to directly impact your dietary intake by affecting perception of flavours and food preferences
\subsubsection{Revealed: The five foods that are key to maintaining good gut health \href{https://www.newscientist.com/article/2383723-revealed-the-five-foods-that-are-key-to-maintaining-good-gut-health/?utm_campaign=RSS%7CNSNS&utm_source=NSNS&utm_medium=RSS&utm_content=home}{\ding{225}}}
\textit{22-Jul-2023}

Fruits and vegetables contain prebiotics, which act as a food source for gut microbes and may boost overall health
\subsubsection{The Saint of Bright Doors review: Fine debut probes nature of memory \href{https://www.newscientist.com/article/mg25934480-700-the-saint-of-bright-doors-review-fine-debut-probes-nature-of-memory/?utm_campaign=RSS%7CNSNS&utm_source=NSNS&utm_medium=RSS&utm_content=home}{\ding{225}}}
\textit{19-Jul-2023}

Stunning sci-fi novel by Vajra Chandrasekera uses magical realism to weave a multi-layered, dreamlike story where the nature of memory and how it can be abused is its deepest theme
\subsection{Phys Org \href{https://phys.org/}{\ding{225}}}
\subsubsection{A chatbot willing to take on questions of all kinds is the latest representation of Jesus for the AI age \href{https://phys.org/news/2023-08-chatbot-kinds-latest-representation-jesus.html}{\ding{225}}}
\textit{02-Aug-2023}

Jesus has been portrayed in many different ways: from a prophet who alerts his audience to the world's imminent end to a philosopher who reflects on the nature of life.
\subsubsection{Dune patterns reveal environmental change on Earth and other planets \href{https://phys.org/news/2023-08-dune-patterns-reveal-environmental-earth.html}{\ding{225}}}
\textit{02-Aug-2023}

Dunes, the mounds of sand formed by the wind that vary from ripples on the beach to towering behemoths in the desert, are incarnations of surface processes, climate change, and the surrounding atmosphere. For decades, scientists have puzzled over why they form different patterns.
\subsubsection{Trump's rise in power resulted from America's racial divide, not cult leadership, study says \href{https://phys.org/news/2023-08-trump-power-resulted-america-racial.html}{\ding{225}}}
\textit{02-Aug-2023}

The power given to Donald Trump is a result of America's racial divide rather than because he is a "cult" leader, a new study says.
\subsubsection{Learning how to control HIV from African genomes \href{https://phys.org/news/2023-08-hiv-african-genomes.html}{\ding{225}}}
\textit{02-Aug-2023}

A study on almost four thousand people of African descent has identified a gene that acts as natural defense against HIV by limiting its replication in certain white blood cells. An international effort co-led by EPFL, Canada's National Microbiology Laboratory, and Imperial College London, it paves the way for new treatment strategies and underscores the importance of studying diverse ancestral populations to better address their specific medical needs and global health disparities.
\subsubsection{Hiring refugees is not just 'doing a good thing': Research shows it can also help businesses \href{https://phys.org/news/2023-08-hiring-refugees-good-businesses.html}{\ding{225}}}
\textit{02-Aug-2023}

The global refugee population is more than 26 million people, according to some estimates. Such largescale movements of people affect many countries and have created significant interest among business and management researchers in recent years as companies try to work out how to successfully integrate refugees into their workforces.
\subsubsection{Fiber optic cables detect and characterize earthquakes \href{https://phys.org/news/2023-08-fiber-optic-cables-characterize-earthquakes.html}{\ding{225}}}
\textit{02-Aug-2023}

In California, thousands of miles of fiber optic cables crisscross the state, providing people with internet. But these underground cables can also have a surprising secondary function: they can sense and measure earthquakes. In a new study at Caltech, scientists report using a section of fiber optic cable to measure intricate details of a magnitude 6 earthquake, pinpointing the time and location of four individual asperities, the "stuck" areas of the fault, that led to the rupture.
\subsubsection{Substitution of tolerant for sensitive species balances ecosystems in agricultural areas, study says \href{https://phys.org/news/2023-08-substitution-tolerant-sensitive-species-ecosystems.html}{\ding{225}}}
\textit{02-Aug-2023}

Contributing to the pursuit of sustainable farming, especially sugarcane growing, Brazilian researchers have shown that water bodies such as ponds and even puddles can maintain ecosystem services, provided there are tolerant animals in the vicinity to replace those most sensitive to agricultural practices.
\subsubsection{Meg 2: The truth about the extinct mega shark, and why even this ridiculous film could inspire future paleontologists \href{https://phys.org/news/2023-08-meg-truth-extinct-mega-shark.html}{\ding{225}}}
\textit{02-Aug-2023}

Otodus megalodon, the biggest shark of all time, has long captured the imaginations of paleontologists and the public alike. Scientific fascination spawns from the sheer enormity of their fossilized teeth. As big as human hands and serrated like kitchen knives, they were used for cutting down whales unlucky enough to encounter these sharks.
\subsubsection{Airbus partners with Voyager Space to build ISS replacement \href{https://phys.org/news/2023-08-airbus-partners-voyager-space-iss.html}{\ding{225}}}
\textit{02-Aug-2023}

Airbus and US space exploration firm Voyager Space announced Wednesday a joint venture to develop Starlab, a commercial alternative to replace the International Space Station (ISS) by the end of the decade.
\subsubsection{Floods for miles: swathes of China underwater after historic rain \href{https://phys.org/news/2023-08-miles-swathes-china-underwater-historic.html}{\ding{225}}}
\textit{02-Aug-2023}

Swathes of northern China were submerged in filthy floodwater on Wednesday after days of historic rainfall battered the capital city of Beijing and surrounding areas.
\subsubsection{Running wild: stray dogs threaten rare Balkan lynx \href{https://phys.org/news/2023-08-wild-stray-dogs-threaten-rare.html}{\ding{225}}}
\textit{02-Aug-2023}

For years, the Balkan lynx has struggled to survive as deforestation destroyed its habitat and poachers targeted the elusive mountain cat along with the animals it relies on for food.
\subsubsection{The heaviest animal ever may be this ancient whale found in the Peruvian desert \href{https://phys.org/news/2023-08-heaviest-animal-ancient-whale-peruvian.html}{\ding{225}}}
\textit{02-Aug-2023}

There could be a new contender for heaviest animal to ever live. While today's blue whale has long held the title, scientists have dug up fossils from an ancient giant that could tip the scales.
\subsubsection{Kuwait's scorching summers a warning for heating planet \href{https://phys.org/news/2023-08-kuwait-summers-planet.html}{\ding{225}}}
\textit{02-Aug-2023}

As the blazing summer sun beats down on Kuwait, shoppers stroll down a promenade lined with palm trees and European-style boutiques, all without breaking a sweat.
\subsubsection{The 2022 Sichuan-Chongqing spatio-temporally compound extremes: A bitter taste of novel hazards \href{https://phys.org/news/2023-08-sichuan-chongqing-spatio-temporally-compound-extremes-bitter.html}{\ding{225}}}
\textit{02-Aug-2023}

A new study led by Dr. Zengchao Hao (College of Water Sciences, Beijing Normal University) and Dr. Yang Chen (Chinese Academy of Meteorological Sciences) documents the unfolding process, reason and impact of compounding and cascading among multiple weather and climate extremes during the course of summer 2022 across the Sichuan Chongqing region.
\subsubsection{Research finds scandals have less impact on politicians than they used to \href{https://phys.org/news/2023-08-scandals-impact-politicians.html}{\ding{225}}}
\textit{02-Aug-2023}

Modern American politics has been plagued by scandals from Watergate to Bill Clinton and Monica Lewinsky, to Donald Trump's Access Hollywood tapes and impeachments. More recently, President Joe Biden's son Hunter faces tax and gun possession charges, casting a shadow over his father's re-election bid.
\subsubsection{Roles of chlorogenic acid in regulating growth performance and immune function of broilers \href{https://phys.org/news/2023-08-roles-chlorogenic-acid-growth-immune.html}{\ding{225}}}
\textit{02-Aug-2023}

Intensive farming practices have gained popularity in recent decades, largely due to the escalated demand for poultry products. Nonetheless, the high stocking densities these methods employ have amplified the susceptibility of commercial broilers to numerous stress factors.
\subsubsection{Is traditional heterosexual romance sexist? \href{https://phys.org/news/2023-08-traditional-heterosexual-romance-sexist.html}{\ding{225}}}
\textit{02-Aug-2023}

Despite progress towards greater gender equality, many people remain stubbornly attached to old-fashioned gender roles in romantic relationships between women and men.
\subsubsection{The way brands address getting called out on Twitter affects their bottom line \href{https://phys.org/news/2023-08-brands-twitter-affects-bottom-line.html}{\ding{225}}}
\textit{02-Aug-2023}

In the digital age, a new Twitter strategy can have implications for a healthy bottom line. How companies handle customer complaints on social media plays a critical role in their customer-focused performance management systems. However, there has been a notable lack of descriptive information related to assessing managerial performance based on the handling of online complaints.
\subsubsection{Unique study shows that wild predators can be trained to hunt alien species they have never seen before \href{https://phys.org/news/2023-08-unique-wild-predators-alien-species.html}{\ding{225}}}
\textit{02-Aug-2023}

Humans have trained domestic animals for thousands of years, to help with farming, transport, or hunting.
\subsubsection{The reaction to 'X,' Elon Musk's rebrand of Twitter, reflects how we feel about brands \href{https://phys.org/news/2023-08-reaction-elon-musk-rebrand-twitter.html}{\ding{225}}}
\textit{02-Aug-2023}

Twitter has long been known for its iconic Blue Bird. On July 23, Elon Musk announced that this famed logo was going to be replaced with an "X." After a series of Musk-driven blunders, the disappearance of the Blue Bird has been seen by some as the final straw in the erasure of Twitter as we know it.
\subsubsection{Novel fatty acids-governed cannibalism in beneficial rhizosphere Bacillus enhances biofilm formation \href{https://phys.org/news/2023-08-fatty-acids-governed-cannibalism-beneficial-rhizosphere.html}{\ding{225}}}
\textit{02-Aug-2023}

When considering a bacterial population as a multicellular community, it is imperative to understand the inherent roles of maintaining homeostasis and viability in response to environmental factors. Cannibalism is a strategy to cope with nutrient deficiency in the environment, which maintains survival of the Bacillus population. Although cannibalism is well-investigated in Bacillus subtilis, the model species of Bacillus, its mechanism remains unknown in other Bacillus species.
\subsubsection{Webb spotlights gravitational arcs in 'El Gordo' galaxy cluster \href{https://phys.org/news/2023-08-webb-spotlights-gravitational-arcs-el.html}{\ding{225}}}
\textit{02-Aug-2023}

A new image of the galaxy cluster known as "El Gordo" is revealing distant and dusty objects never seen before, and providing a bounty of fresh science. The infrared image, taken by NASA's James Webb Space Telescope, displays a variety of unusual, distorted background galaxies that were only hinted at in previous Hubble Space Telescope images.
\subsubsection{How social media can spread conspiracy theories and even spark violence \href{https://phys.org/news/2023-08-social-media-conspiracy-theories-violence.html}{\ding{225}}}
\textit{02-Aug-2023}

Conspiracy theory beliefs and (more generally) misinformation may be groundless, but they can have a range of harmful real-world consequences, including spreading lies, undermining trust in media and government institutions and inciting violent or even extremist behaviors.
\subsubsection{Nuclear war would be more devastating for Earth's climate than cold war predictions, even with fewer weapons \href{https://phys.org/news/2023-08-nuclear-war-devastating-earth-climate.html}{\ding{225}}}
\textit{02-Aug-2023}

Christopher Nolan's biopic of J. Robert Oppenheimer has revived morbid curiosity in the destructive power of nuclear weapons. There are now an estimated 12,512 nuclear warheads.
\subsubsection{Climate change contributes to violence against children. Here's how \href{https://phys.org/news/2023-08-climate-contributes-violence-children.html}{\ding{225}}}
\textit{02-Aug-2023}

Every day of the northern hemisphere's summer in 2023 seems to bring a calamitous headline about the climate: heat waves, wildfires, massive hailstorms.
\subsubsection{An ancient grain unlocks genetic secrets for making bread wheat more resilient \href{https://phys.org/news/2023-08-ancient-grain-genetic-secrets-bread.html}{\ding{225}}}
\textit{02-Aug-2023}

Building on the Middle East's reputation as one of the historical birthplaces of cereal crop domestication, a KAUST-led team has compiled the first complete genome map of an ancient grain known as einkorn.
\subsubsection{My Climate View provides farmers insights into future climate based on location and commodity \href{https://phys.org/news/2023-08-climate-view-farmers-insights-future.html}{\ding{225}}}
\textit{02-Aug-2023}

It's around this time of year that Australian cherry growers look for cooler days.
\subsubsection{Nature's kitchen: How a chemical reaction used by cooks helped create life on Earth \href{https://phys.org/news/2023-08-nature-kitchen-chemical-reaction-cooks.html}{\ding{225}}}
\textit{02-Aug-2023}

A chemical process used in the browning of food to give it its distinct smell and taste is probably happening deep in the oceans, where it helped create the conditions necessary for life.
\subsubsection{Researchers prefer same-gender co-authors, study confirms \href{https://phys.org/news/2023-08-same-gender-co-authors.html}{\ding{225}}}
\textit{02-Aug-2023}

Researchers are more likely to pen scientific papers with co-authors of the same gender, a pattern that cannot be simply explained by the varying gender representation across scientific disciplines and time, according to joint research from Cornell and the University of Washington.
\subsubsection{Expert explains how cities can beat the heat by building better \href{https://phys.org/news/2023-08-expert-cities.html}{\ding{225}}}
\textit{02-Aug-2023}

The United Nations recently declared that the world is now in an "era of global boiling."
\subsection{Scientific American \href{https://www.scientificamerican.com/}{\ding{225}}}
\subsubsection{Bizarre-Looking Colossus Whale May Have Been Heaviest Animal Ever (Sorry, Blue Whales) \href{https://www.scientificamerican.com/article/bizarre-looking-colossus-whale-may-have-been-heaviest-animal-ever-sorry-blue-whales/}{\ding{225}}}
\textit{02-Aug-2023}

\&ldquo;I\&rsquo;ve never seen anything like it,\&rdquo; says a paleontologist not involved in the discovery of a 40-million-year-old fossilized whale
\subsubsection{Science, Destroyer of Worlds--And Movie Scripts \href{https://www.scientificamerican.com/article/science-destroyer-of-worlds-and-movie-scripts/}{\ding{225}}}
\textit{02-Aug-2023}

<em>Oppenheimer</em> won\&rsquo;t bomb in the box office, but despite its director\&rsquo;s best efforts, the science in the film is a bit of a fizzle
\subsubsection{Social Media Is Rewriting the Banking Playbook \href{https://www.scientificamerican.com/article/social-media-is-rewriting-the-banking-playbook/}{\ding{225}}}
\textit{02-Aug-2023}

Our financial systems need to be overhauled to avoid the kind of social-media-fueled bank run that killed Silicon Valley Bank
\subsubsection{Yes, AI Models Can Get Worse over Time \href{https://www.scientificamerican.com/article/yes-ai-models-can-get-worse-over-time/}{\ding{225}}}
\textit{02-Aug-2023}

More training and more data can have unintended consequences for machine-learning models such as GPT-4
\subsubsection{Could Weight-Loss Drugs Curb Addiction? Your Health, Quickly, Episode 12 \href{https://www.scientificamerican.com/podcast/episode/could-weight-loss-drugs-curb-addiction-your-health-quickly-episode-12/}{\ding{225}}}
\textit{02-Aug-2023}

Drugs such as Wegovy and Ozempic might help people tackle substance abuse as well as shed pounds.
\subsubsection{'Virgin Birth' Engineered into Female Animals for First Time \href{https://www.scientificamerican.com/article/virgin-birth-engineered-into-female-animals-for-first-time/}{\ding{225}}}
\textit{01-Aug-2023}

Scientists altered the genomes of female fruit flies, allowing them to reproduce without any contribution from a male
\subsubsection{NASA Detects 'Heartbeat' from Voyager 2 Spacecraft after Losing Contact \href{https://www.scientificamerican.com/article/nasa-detects-heartbeat-from-voyager-2-spacecraft-after-losing-contact/}{\ding{225}}}
\textit{01-Aug-2023}

A glitch may have silenced NASA\&rsquo;s Voyager 2 spacecraft until mid-October\&mdash;but a \&ldquo;heartbeat\&rdquo; signal offers hope for reestablishing contact earlier
\subsubsection{A Sun Shield over Earth? Catch an Asteroid, and It Might Work \href{https://www.scientificamerican.com/article/a-sun-shield-over-earth-catch-an-asteroid-and-it-might-work/}{\ding{225}}}
\textit{01-Aug-2023}

A resurfaced idea for solar geoengineering imagines a sunlight-blocking space shield tethered to an asteroid
\subsubsection{Science Corrects Itself, Right? A Scandal at Stanford Says It Doesn't \href{https://www.scientificamerican.com/article/science-corrects-itself-right-a-scandal-at-stanford-says-it-doesnt/}{\ding{225}}}
\textit{01-Aug-2023}

What does it take to correct the scientific record? And who\&mdash;and what\&mdash;stands in the way? The answer to both questions is: everyone
\subsubsection{Space Debris Will Block Space Exploration unless We Start Acting Sustainably \href{https://www.scientificamerican.com/article/space-debris-will-block-space-exploration-unless-we-start-acting-sustainably/}{\ding{225}}}
\textit{01-Aug-2023}

We need satellites and rocket bodies designed with an end-of-life plan to keep space uncluttered and navigable
\subsubsection{The Ukraine War Is an Environmental Catastrophe with Global Consequences \href{https://www.scientificamerican.com/article/the-ukraine-war-is-an-environmental-catastrophe-with-global-consequences/}{\ding{225}}}
\textit{01-Aug-2023}

When it\&rsquo;s time to rebuild, we must prioritize more sustainable and resilient infrastructure in Ukraine
\subsubsection{Your Genes May Influence What You Like to Eat \href{https://www.scientificamerican.com/article/your-genes-may-influence-what-you-like-to-eat/}{\ding{225}}}
\textit{01-Aug-2023}

New research identifies genome areas linked to dietary patterns and our taste for things such as tea, tobacco and grapes
\subsubsection{Unregulated AI Will Worsen Inequality, Warns Nobel-Winning Economist Joseph Stiglitz \href{https://www.scientificamerican.com/article/unregulated-ai-will-worsen-inequality-warns-nobel-winning-economist-joseph-stiglitz/}{\ding{225}}}
\textit{01-Aug-2023}

A Nobel laureate in economics explains how artificial intelligence will affect inequality\&mdash;and how solutions such as a shorter work week might mitigate its negative effects
\subsubsection{U.S. Looks to Mongolia, Wedged between China and Russia, for Critical Minerals \href{https://www.scientificamerican.com/article/u-s-looks-to-mongolia-wedged-between-china-and-russia-for-critical-minerals/}{\ding{225}}}
\textit{31-Jul-2023}

All routes out of the landlocked country touch China or Russia, presenting diplomatic and physical challenges
\subsubsection{How to Roll a Joint Perfectly, according to Science \href{https://www.scientificamerican.com/podcast/episode/how-to-roll-a-joint-perfectly-according-to-science/}{\ding{225}}}
\textit{31-Jul-2023}

Scientists used a smoking machine\&mdash;complete with a 3-D-printed mouthpiece\&mdash;to figureout how to get the most cannabinoid per puff.
\subsubsection{Does Barbie Affect Body Image? What the Science Shows \href{https://www.scientificamerican.com/article/does-barbie-affect-body-image-what-the-science-shows/}{\ding{225}}}
\textit{31-Jul-2023}

A clinical health psychologist talks about Barbie\&rsquo;s influence on how women and girls view theirbody
\subsubsection{These Salamanders Steal Genes and Can Have up to Five Extra Sets of Chromosomes \href{https://www.scientificamerican.com/video/these-salamanders-steal-genes-and-can-have-up-to-five-extra-sets-of-chromosomes/}{\ding{225}}}
\textit{31-Jul-2023}

Unisexual salamanders in the genus <em>Ambystoma</em> appear to be the only creatures in theworld that reproduce the way they do. Researchers know how, but the why is still being figured out.
\subsubsection{Art May Be in the Body of the Beholder \href{https://www.scientificamerican.com/article/art-may-be-in-the-body-of-the-beholder/}{\ding{225}}}
\textit{31-Jul-2023}

A study suggests a complex interplay between bodily feeling, emotion and art
\subsubsection{Did Earth's Water Come from Meteorites? \href{https://www.scientificamerican.com/article/did-earths-water-come-from-meteorites/}{\ding{225}}}
\textit{31-Jul-2023}

At least some of our planet\&rsquo;s water was carried here by hydrogen-rich space rocks, but it\&rsquo;s not yet clear how much
\subsubsection{Discovery of Elusive 'Einstein' Tile Raises More Questions Than It Answers \href{https://www.scientificamerican.com/article/discovery-of-elusive-einstein-tile-raises-more-questions-than-it-answers/}{\ding{225}}}
\textit{31-Jul-2023}

A surprisingly simple answer to a mathematical puzzle intrigues the math world
\subsubsection{Electric Bandages Heal Wounds That Won't Close, Animal Study Shows \href{https://www.scientificamerican.com/article/electric-bandages-heal-wounds-that-wont-close-animal-study-shows/}{\ding{225}}}
\textit{31-Jul-2023}

New technology combines electricity and drugs to stimulate healing of tenacious wounds
\subsubsection{How 'Zombie' Fires Rise from the Dead in Spring \href{https://www.scientificamerican.com/article/how-zombie-fires-rise-from-the-dead-in-spring/}{\ding{225}}}
\textit{31-Jul-2023}

As \&ldquo;zombie\&rdquo; fires become more common, new research shows they arise from an unexpected source
\subsubsection{How Wasted Food Turns into Huge Amounts of Greenhouse Gas \href{https://www.scientificamerican.com/article/how-wasted-food-turns-into-huge-amounts-of-greenhouse-gas/}{\ding{225}}}
\textit{31-Jul-2023}

Here\&rsquo;s how food loss and waste threaten the planet at every stage, from harvest to consumption
\subsubsection{Humans Can Spot Tiny Numerical Differences \href{https://www.scientificamerican.com/article/humans-can-spot-tiny-numerical-differences/}{\ding{225}}}
\textit{31-Jul-2023}

Where is the line between knowing and guessing?
\subsubsection{Munching Bugs Gave the First Mammals an Edge \href{https://www.scientificamerican.com/article/munching-bugs-gave-the-first-mammals-an-edge/}{\ding{225}}}
\textit{31-Jul-2023}

Early mammals got ahead by eating insects
\subsection{Science News \href{https://www.sciencenews.org/}{\ding{225}}}
\subsubsection{A colossal ancient whale could be the heaviest animal ever known \href{https://www.sciencenews.org/article/colossal-ancient-whale-heaviest-animal}{\ding{225}}}
\textit{02-Aug-2023}

<em>Perucetus colossus</em> may have tipped the scales at up to 340 metric tons, but some scientists are skeptical it could have sustained that mass.
\subsubsection{July 2023 nailed an unfortunate world record: hottest month ever recorded \href{https://www.sciencenews.org/article/july-2023-just-unfortunate-world-record-hot}{\ding{225}}}
\textit{02-Aug-2023}

Roughly 6.5 billion people, or 4 out of 5 humans, felt the touch of climate change via hotter temperatures during July.
\subsubsection{The newfound Los Angeles thread millipede is ready for its close-up \href{https://www.sciencenews.org/article/new-los-angeles-thread-millipede}{\ding{225}}}
\textit{01-Aug-2023}

Found in Southern California, <em>Illacme socal</em> is the third of its genus found in North America, with the rest of its relatives scattered around the world.
\subsubsection{50 years ago, scientists thought they had found Earth’s oldest rocks \href{https://www.sciencenews.org/article/50-years-ago-earth-oldest-rocks}{\ding{225}}}
\textit{31-Jul-2023}

Even older rocks and minerals continue fueling debates over Earth’s crust, plate tectonics and even when life arose.
\subsubsection{‘Blight’ warns that a future pandemic could start with a fungus \href{https://www.sciencenews.org/article/blight-future-pandemic-fungus-book}{\ding{225}}}
\textit{30-Jul-2023}

‘The Last of Us’ is fiction, but the health dangers posed by fungi are real, a new book explains.
\subsubsection{Playful behavior in rats is controlled by a specific area of their brains \href{https://www.sciencenews.org/article/playful-behavior-rats-area-brain}{\ding{225}}}
\textit{28-Jul-2023}

Cells in a brain region called the periaqueductal gray are activated by chasing and tickling, a study finds. Blocking their activity reduces play in rats.
\subsubsection{Cow poop emits climate-warming methane. Adding red algae may help \href{https://www.sciencenews.org/article/cow-poop-climate-warming-methane-red-algae}{\ding{225}}}
\textit{28-Jul-2023}

Adding a type of methane-inhibiting red algae directly to cow feces cut down methane emission from the poop by about 44 percent, researchers report.
\subsubsection{How geometry solves architectural problems for bees and wasps \href{https://www.sciencenews.org/article/geometry-architectural-problem-bee-wasp}{\ding{225}}}
\textit{27-Jul-2023}

Adding five - and seven - sided cells in pairs during nest building helps the colonyfit together differently sized hexa
gonal cells
, a new study shows.
\subsubsection{The oldest known horseback riding saddle was found in a grave in China \href{https://www.sciencenews.org/article/oldest-saddle-horseback-riding-china-grave}{\ding{225}}}
\textit{27-Jul-2023}

The well-used saddle, dated to more than 2,400 years ago, displays skilled leather- and needlework. Its placement suggests its owner was on a final ride.
\subsubsection{Many sports supplements have no trace of their key ingredients \href{https://www.sciencenews.org/article/sport-supplements-ingredients-dietary}{\ding{225}}}
\textit{26-Jul-2023}

A chemical analysis of 57 supplements found that 40 percent had undetectable amounts of key ingredients. Only 11 percent had accurate amounts.
\subsubsection{The most intense sunlight on Earth can be found in the Atacama Desert \href{https://www.sciencenews.org/article/most-intense-sunlight-earth-atacama-desert}{\ding{225}}}
\textit{26-Jul-2023}

On the Chilean Altiplano plateau, every square meter of the ground receives, on average, more solar power than Mount Everest and occasionally almost as much as Venus.
\subsubsection{Some African birds follow nomadic ants to their next meal \href{https://www.sciencenews.org/article/birds-driver-ants-africa-ecology}{\ding{225}}}
\textit{25-Jul-2023}

Specialized interactions between birds and driver ants in Africa could help explain why the birds are especially sensitive to forest disturbances.
\subsubsection{Here’s how much climate change increases the odds of brutally hot summers \href{https://www.sciencenews.org/article/climate-change-heat-wave-summer}{\ding{225}}}
\textit{25-Jul-2023}

Climate change made 2023’s record-breaking heat waves in the United States, Mexico, China and southern Europe much more likely, new simulations show.
\subsubsection{The James Webb telescope may have spotted stars powered by dark matter \href{https://www.sciencenews.org/article/james-webb-telescope-stars-dark-matter}{\ding{225}}}
\textit{24-Jul-2023}

Three objects in the distant universe bear signs of hypothesized “dark stars,” researchers claim, though others say more definitive data are needed.
\subsubsection{Human embryo replicas have gotten more complex. Here’s what you need to know \href{https://www.sciencenews.org/article/human-embryo-replica-complex-advanced}{\ding{225}}}
\textit{24-Jul-2023}

Lab-engineered human embryo models created from stem cells provide a look at development beyond the first week. But they raise ethical questions.
\subsubsection{How an ancient solar flare illuminated the start of the Viking Age \href{https://www.sciencenews.org/article/viking-age-ancient-solar-flare-trade-archaeology}{\ding{225}}}
\textit{23-Jul-2023}

Improved radiocarbon dating aided by a solar flare in the year 775 sheds light on the early days of Vikings and global trading in medieval times.
\subsubsection{NASA’s DART mission lofted a swarm of boulders into space \href{https://www.sciencenews.org/article/nasa-dart-mission-boulder-asteroid-space}{\ding{225}}}
\textit{21-Jul-2023}

Hubble telescope images of the asteroid Dimorphos reveal a halo of 37 dim, newfound objects — most likely boulders shaken loose from the surface.
\subsubsection{With a new body mapping technique, mouse innards glow with exquisite detail \href{https://www.sciencenews.org/article/new-body-mapping-technique-mouse-organ}{\ding{225}}}
\textit{21-Jul-2023}

Removing cholesterol from mouse bodies lets fluorescently labeled proteins infiltrate every tissue, helping researchers to map entire body systems.
\subsubsection{‘The Next Supercontinent’ predicts a future collision of North America and Asia \href{https://www.sciencenews.org/article/next-supercontinent-book-north-america-asia-collision}{\ding{225}}}
\textit{21-Jul-2023}

In his new book, Ross Mitchell traces the dance of the continents through time to predict what Amasia, the next supercontinent, might look like.
\subsubsection{Time in nature or exercise is touted for happiness. But evidence is lacking \href{https://www.sciencenews.org/article/nature-exercise-happiness-psychology}{\ding{225}}}
\textit{20-Jul-2023}

A review of hundreds of studies finds limited strong scientific evidence to support many common recommendations for leading a happier life.
